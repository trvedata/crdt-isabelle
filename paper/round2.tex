\documentclass{letter}
\usepackage[margin=1in]{geometry}
\usepackage{url}
\address{OOPSLA Paper \#94: \\ Verifying Strong Eventual Consistency \\ in Distributed Systems}
\signature{The authors}
\begin{document}
\begin{letter}{}
\opening{Dear reviewers,}

Thank you for your helpful comments on our paper in the first round.
We enclose a revised version and trust that it addresses your comments.
Our changes compared to the last round are as follows:

\begin{itemize}
\item Our full proof is now published on the Isabelle Archive of Formal Proofs (\url{https://www.isa-afp.org/entries/CRDT.shtml}), and we have referenced it from the paper.
\item As requested by reviewer A, we now back up our claim of flawed proofs of SEC algorithms in the introduction by including a literature reference and a forward reference to Section 8.
\item As requested by reviewer B, we included a brief explanation of the distinction between the $\longrightarrow$ and $\Longrightarrow$ arrows in Isabelle at the bottom of page 5.
It is difficult to explain concisely, because the distinction amounts essentially to an implementation detail of Isabelle, and a full explanation would require several pages of digression into Isabelle's internals.
However, we hope that the brief explanation will nevertheless be useful.
\item Addressing a question by reviewer A, we added the second paragraph of Section 5.2 to explain why our network uses a broadcast abstraction, and to outline how it is often implemented in practice (as an overlay network atop unicast TCP).
\item Reviewer B also suggested discussing the robustness of algorithms to minor variations in infrastructure assumptions, which we have done briefly in the penultimate paragraph of Section 5, on page 14.
\item In Section 8.1 we added the 4th, 5th and 6th paragraphs (top half of page 22) to explain the $\mathit{TP}_1$ and $\mathit{TP}_2$ properties, as requested by reviewer A.
\item Following on from reviewer A's comment on Jepsen testing, we added a final paragraph to Section 9 to explain that the purpose of generating executable OCaml code is not to perform an empirical evaluation, but merely to demonstrate that we have not used any uncomputable functions.
\item We have made the inline Isabelle code snippets easier to read by making the presentation neater (indentation, spacing, line breaks, font size).
The content of the snippets is unchanged.
\item Improved clarity through minor (semantically neutral) rewordings throughout the paper.
\end{itemize}

\closing{Regards,}
\end{letter}
\end{document}
