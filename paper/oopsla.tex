\documentclass[acmlarge,review,anonymous]{acmart}\settopmatter{printfolios=true}

\bibliographystyle{ACM-Reference-Format}
\citestyle{acmauthoryear}
\usepackage[english]{babel}

\usepackage{isabelle,isabellesym}
\isabellestyle{it}

\setcopyright{none} % For review submission

\begin{document}
\title{Formal Verification of Peer-to-Peer Collaborative Editing}
%\author{Victor~B.~F.~Gomes, Martin Kleppmann, Dominic P.~Mulligan,\\Alastair R. Beresford}
%\date{Computer Laboratory, University of Cambridge}

\maketitle

\begin{abstract}
To be completed...
\end{abstract}

%%%%%%%%%%%%%%%%%%%%%%%%%%%%%%%%%%%%%%%%%%%%%%%%%%%%%%%%%%%%%%%%%%%%%%%%%%%%%%%%
% Section
%%%%%%%%%%%%%%%%%%%%%%%%%%%%%%%%%%%%%%%%%%%%%%%%%%%%%%%%%%%%%%%%%%%%%%%%%%%%%%%%

\section{Introduction}
\label{sect.introduction}

Collaborative editing applications such as Google Docs~\cite{DayRichter:2010tt}, Microsoft Word
Online, Etherpad~\cite{Etherpad:2011um}, and Novell Vibe~\cite{Spiewak:2010vw} are increasing in
popularity. A common feature of these tools is that they allow several users to concurrently modify
a document without having to send the document back and forth (e.g., by email), and without
requiring any exclusive locking or manual resolution of merge conflicts.

However, all currently deployed collaborative editing systems rely on a central server to determine
a sequential order of edit operations, a design originally pioneered by the Jupiter
system~\cite{Nichols:1995fd}. This architecture has the advantage of simplifying the collaborative
editing algorithm by restricting the concurrency in the system, but it has the downside of placing
significant trust in that single server: it is at risk of being compromised, censored, seized, or
otherwise subverted by adversaries. For sensitive scenarios, such as communication between
dissidents of a repressive regime, such centralization is problematic.

Decentralized peer-to-peer systems with end-to-end encryption can be more robust against such
interference. In this setting, especially when the participating nodes are mobile devices, totally
ordered broadcast of editing operations is prohibitively expensive~\cite{Attiya:2015dm}. Thus, there
has been significant interest in collaborative editing algorithms that work correctly in the face of
the increased concurrency encountered in peer-to-peer systems~\cite{Randolph:2015gj}.

There are two families of algorithms for collaborative editing: \emph{operational transformation}
(OT)~\cite{Ellis:1989ue,Ressel:1996wx,Oster:2006tr,Sun:1998vf,Sun:1998un,Suleiman:1998eu,Nichols:1995fd}
and \emph{conflict-free replicated datatypes}
(CRDTs)~\cite{Shapiro:2011wy,Roh:2011dw,Preguica:2009fz,Oster:2006wj,Weiss:2010hx,Nedelec:2013ky,Kleppmann:2016ve}.
Both allow a document to be modified concurrently on different replicas, with changes applied
immediately to the local copy, while asynchronously propagating changes to other replicas. The
goal of these algorithms is to ensure that for all concurrent executions, the replicas converge
toward the same state without any edits being lost (a property known as \emph{strong eventual
consistency}~\cite{Shapiro:2011un}).

However, these algorithms have a checkered history. OT algorithms have a reputation of being very
difficult to understand and to implement correctly~\cite{Spiewak:2010vw}. Despite the fact that OT
has been studied for almost three decades~\cite{Ellis:1989ue}, few algorithms work correctly in a
peer-to-peer setting, and several published algorithms were later shown to violate their supposed
convergence guarantees~\cite{Imine:2003ks,Imine:2006kn}. It has even been proved that in the classic
formulation of OT it is impossible to achieve the $\mathit{TP}_2$ property required for convergence
in a peer-to-peer setting, and that additional parameters must be added to transformation functions
to make convergence possible~\cite{Randolph:2015gj}.

CRDTs are a more recent development~\cite{Shapiro:2011un}. While OT is based on transforming
non-commutative operations so that they have the same effect when reordered, CRDTs define operations
in a way that makes them commutative by design, making them more amenable to peer-to-peer settings
in which each node may apply edits in a different order. CRDTs also have attractive performance
characteristics~\cite{Mehdi:2011ke}.

To date there has been fairly little formal verification of the correctness of CRDTs, and the
history of broken OT algorithms highlights the inadequacy of informal reasoning in this domain. In
this work we contribute to the formal basis of collaborative editing algorithms by using the
interactive proof assistant software Isabelle (TODO citation) to develop machine-checked proofs of
correctness for CRDTs.

In particular, we study the Replicated Growable Array (RGA) CRDT~\cite{Roh:2011dw}, which represents
a collaboratively editable document as a sequence of characters. There are previous pen-and-paper
correctness proofs of RGA in the literature~\cite{Attiya:2016kh,Kleppmann:2016ve,Roh:2009ws}, but to
our knowledge, ours is the first mechanized proof of the convergence of RGA. The algorithm is
subtle~-- Attiya et al.\ recently wrote, ``the reason why RGA actually works has been a bit of a
mystery''~\cite{Attiya:2016kh}~-- which makes formal verification particularly important.

Our proof is structured in four modules:
\begin{enumerate}
    \item A general convergence theorem that applies in any system where concurrent operations are
        commutative;
    \item A formal model of a network protocol providing reliable, causally-ordered broadcast;
    \item An implementation of the RGA algorithm, and a proof that well-formed, concurrent insertion
        and deletion operations commute;
    \item A proof that when the RGA algorithm is executed in our network model, all possible
        executions are well-formed, and thus converge.
\end{enumerate}

In doing so, we go much further than previous correctness proofs of collaborative editing
algorithms. Previous formalizations of OT using theorem
provers~\cite{Imine:2003ks,Imine:2006kn,Sinchuk:2016cf,Jungnickel:2015ua} focus on proving that the
transformation functions satisfy given properties (such as the transformation properties
$\mathit{TP}_1$ and $\mathit{TP}_2$~\cite{Oster:2006tr,Ressel:1996wx}), and do not explicitly model
the network. A previous formalization of CRDTs~\cite{Zeller:2014fl}, also using Isabelle, considers
other datatypes (sets, registers, counters) but not the ordered sequence datatype provided by RGA.

By including a model of the network in our proof, we rule out a larger set of potential errors in
the algorithm that may result from the interaction of operation properties with assumptions about
the network. Moreover, our network model and convergence theorems are independent of any particular
CRDT, so they can be reused for correctness proofs of any other replicated datatype that is based on
operation commutativity, encompassing a wide range of CRDTs~\cite{Baquero:2014ed}.

Besides presenting the first machine-checked proof of the RGA algorithm, our main contribution in
this paper is to establish a modular toolkit of proof techniques and building blocks for
machine-checked correctness proofs of operation-based CRDTs. Our proofs are broken down into modules
with well-defined properties, allowing modules to be reused for proofs of new datatypes in future.
By making formal verification easier, we hope to provide a strong foundation for the development of
the next generation of algorithms for collaborative editing.


\section{Background}
\label{sect.background}

\section{Strong Eventual Consistency}
\label{sect.convergence}

\emph{Strong eventual consistency} is a safety property that guarantees that any
two given nodes in a network that have received the same set of events converge to the same state.
We prove this convergence property in an abstract setting, that is, any two arbitrary list of distinct events, enriched by an order relation that satisfies a \emph{consistency property}, converge.

Let $\mathit{Events}$ be the set of events enriched with a preorder $\prec$. We
write $x \prec y$ and say that the event $x$ \emph{happens before} the event
$y$.  Moreover we assume that there exists an interpretation partial function
$\langle\_\rangle$ that maps events to \emph{operations}, which are \emph{state
transformers} on a set of states $\Sigma$.  This is implemented in Isabelle
by the following \textbf{locale} statement.

\begin{isabellebody}
  \isanewline
\isacommand{locale}\isamarkupfalse%
\ happens{\isacharunderscore}before\ {\isacharequal}\ preorder\ hb{\isacharunderscore}weak\ hb\isanewline
\ \ \isakeyword{for}\ hb{\isacharunderscore}weak\ {\isacharcolon}{\isacharcolon}\ {\isachardoublequoteopen}{\isacharprime}a\ {\isasymRightarrow}\ {\isacharprime}a\ {\isasymRightarrow}\ bool{\isachardoublequoteclose}\ \ {\isacharparenleft}\isakeyword{infix}\ {\isachardoublequoteopen}{\isasympreceq}{\isachardoublequoteclose}\ {\isadigit{5}}{\isadigit{0}}{\isacharparenright}\isanewline
\ \ \isakeyword{and}\ hb\ {\isacharcolon}{\isacharcolon}\ {\isachardoublequoteopen}{\isacharprime}a\ {\isasymRightarrow}\ {\isacharprime}a\ {\isasymRightarrow}\ bool{\isachardoublequoteclose}\ \ \ \ \ \ \ {\isacharparenleft}\isakeyword{infix}\ {\isachardoublequoteopen}{\isasymprec}{\isachardoublequoteclose}\ {\isadigit{5}}{\isadigit{0}}{\isacharparenright}\ {\isacharplus}\isanewline
\ \ \isakeyword{fixes}\ interp\ {\isacharcolon}{\isacharcolon}\ {\isachardoublequoteopen}{\isacharprime}a\ {\isasymRightarrow}\ {\isacharprime}b\ {\isasymrightharpoonup}\ {\isacharprime}b{\isachardoublequoteclose}\ {\isacharparenleft}{\isachardoublequoteopen}{\isasymlangle}{\isacharunderscore}{\isasymrangle}{\isachardoublequoteclose}\ {\isacharbrackleft}{\isadigit{0}}{\isacharbrackright}\ {\isadigit{1}}{\isadigit{0}}{\isadigit{0}}{\isadigit{0}}{\isacharparenright}\isanewline
\end{isabellebody}

We say that two events (or operations) $x$ and $y$ are \emph{concurrent}
whenever one does not happens before the other, that is, $\neg (x \prec y)$ and
$\neg (y \prec x)$.

\begin{isabellebody}
  \isanewline
\isacommand{definition}\isamarkupfalse%
\ concurrent\ {\isacharcolon}{\isacharcolon}\ {\isachardoublequoteopen}{\isacharprime}a\ {\isasymRightarrow}\ {\isacharprime}a\ {\isasymRightarrow}\ bool{\isachardoublequoteclose}\ {\isacharparenleft}\isakeyword{infix}\ {\isachardoublequoteopen}{\isasymparallel}{\isachardoublequoteclose}\ {\isadigit{5}}{\isadigit{0}}{\isacharparenright}\ \isakeyword{where}\isanewline
\ \ {\isachardoublequoteopen}x\ {\isasymparallel}\ y\ {\isasymequiv}\ {\isasymnot}\ {\isacharparenleft}x\ {\isasymprec}\ y{\isacharparenright}\ {\isasymand}\ {\isasymnot}\ {\isacharparenleft}y\ {\isasymprec}\ x{\isacharparenright}{\isachardoublequoteclose}\isanewline
\end{isabellebody}

A list of events $xs$ is \emph{hb-consistent}, or simply \emph{consistent},
whenever for any element $y$, it is not the case that $y$ happens before $x$
for all $x$ that appears before $y$ in the list. That is, $x$ and $y$ are
either concurrent or $x$ happens before $y$.  This is implemented in Isabelle
by an inductive predicate.

\begin{isabellebody}
  \isanewline
\isacommand{inductive}\isamarkupfalse%
\ hb{\isacharunderscore}consistent\ {\isacharcolon}{\isacharcolon}\ {\isachardoublequoteopen}{\isacharprime}a\ list\ {\isasymRightarrow}\ bool{\isachardoublequoteclose}\ \isakeyword{where}\isanewline
\ \ {\isachardoublequoteopen}hb{\isacharunderscore}consistent\ {\isacharbrackleft}{\isacharbrackright}{\isachardoublequoteclose}\ {\isacharbar}\isanewline
\ \ {\isachardoublequoteopen}{\isasymlbrakk}\ hb{\isacharunderscore}consistent\ xs{\isacharsemicolon}\ {\isasymforall}x\ {\isasymin}\ set\ xs{\isachardot}\ {\isasymnot}\ y\ {\isasymprec}\ x\ {\isasymrbrakk}\ {\isasymLongrightarrow}\ hb{\isacharunderscore}consistent\ {\isacharparenleft}xs\ {\isacharat}\ {\isacharbrackleft}y{\isacharbrackright}{\isacharparenright}{\isachardoublequoteclose}\isanewline
\end{isabellebody}

Additionally, we say that concurrent operations commute in a list $xs$ whenever
for all concurrent events $x$ and $y$ in the list $xs$, the Kleisli arrow
composition of their operations commute $\langle x \rangle \rhd \langle y
\rangle = \langle y \rangle \rhd \langle x \rangle$.

\begin{isabellebody}
  \isanewline
\isacommand{definition}\isamarkupfalse%
\ concurrent{\isacharunderscore}ops{\isacharunderscore}commute\ {\isacharcolon}{\isacharcolon}\ {\isachardoublequoteopen}{\isacharprime}a\ list\ {\isasymRightarrow}\ bool{\isachardoublequoteclose}\ \isakeyword{where}\isanewline
\ \ {\isachardoublequoteopen}concurrent{\isacharunderscore}ops{\isacharunderscore}commute\ xs\ {\isasymequiv}
\ {\isasymforall}x\ y{\isachardot}\ {\isacharbraceleft}x{\isacharcomma}\ y{\isacharbraceright}\ {\isasymsubseteq}\ set\ xs\ {\isasymlongrightarrow}\ x\ {\isasymparallel}\ y\ {\isasymlongrightarrow}\ {\isasymlangle}x{\isasymrangle}{\isasymrhd}{\isasymlangle}y{\isasymrangle}\ {\isacharequal}\ {\isasymlangle}y{\isasymrangle}{\isasymrhd}{\isasymlangle}x{\isasymrangle}{\isachardoublequoteclose}\isanewline

\isacommand{definition}\isamarkupfalse%
\ kleisli\ {\isacharcolon}{\isacharcolon}\ {\isachardoublequoteopen}{\isacharparenleft}{\isacharprime}b\ {\isasymRightarrow}\ {\isacharprime}b\ option{\isacharparenright}\ {\isasymRightarrow}\ {\isacharparenleft}{\isacharprime}b\ {\isasymRightarrow}\ {\isacharprime}b\ option{\isacharparenright}\ {\isasymRightarrow}\ {\isacharparenleft}{\isacharprime}b\ {\isasymRightarrow}\ {\isacharprime}b\ option{\isacharparenright}{\isachardoublequoteclose}\ {\isacharparenleft}\isakeyword{infixr}\ {\isachardoublequoteopen}{\isasymrhd}{\isachardoublequoteclose}\ {\isadigit{6}}{\isadigit{5}}{\isacharparenright}\ \isakeyword{where}\isanewline
\ \ {\isachardoublequoteopen}f\ {\isasymrhd}\ g\ {\isasymequiv}\ {\isasymlambda}x{\isachardot}\ f\ x\ {\isasymbind}\ {\isacharparenleft}{\isasymlambda}fx{\isachardot}\ g\ fx{\isacharparenright}{\isachardoublequoteclose}\isanewline
\end{isabellebody}

Given a list of events $xs$, \emph{apply-operations} $xs$ is the state
transformer equivalent of applying each operation associated to the events in
the list. 

\begin{isabellebody}
  \isanewline
\isacommand{definition}\isamarkupfalse%
\ apply{\isacharunderscore}operations\ {\isacharcolon}{\isacharcolon}\ {\isachardoublequoteopen}{\isacharprime}a\ list\ {\isasymRightarrow}\ {\isacharprime}b\ {\isasymrightharpoonup}\ {\isacharprime}b{\isachardoublequoteclose}\ \isakeyword{where}\isanewline
\ \ {\isachardoublequoteopen}apply{\isacharunderscore}operations\ es\ s\ {\isasymequiv}\ {\isacharparenleft}foldl\ {\isacharparenleft}op\ {\isasymrhd}{\isacharparenright}\ Some\ {\isacharparenleft}map\ interp\ es{\isacharparenright}{\isacharparenright}\ s{\isachardoublequoteclose}\isanewline
\end{isabellebody}

We can now formally state the abstract convergence theorem: two consistent list of distinct events in which concurrent operations commute that have the same set of elements yields the same state transformer.

\begin{isabellebody}
  \isanewline
\isacommand{theorem}\isamarkupfalse%
\ \ convergence{\isacharcolon}\isanewline
\ \ \isakeyword{assumes}\ {\isachardoublequoteopen}set\ xs\ {\isacharequal}\ set\ ys{\isachardoublequoteclose}\isanewline
\ \ \ \ \ \ \ \ \ \ {\isachardoublequoteopen}concurrent{\isacharunderscore}ops{\isacharunderscore}commute\ xs{\isachardoublequoteclose}\isanewline
\ \ \ \ \ \ \ \ \ \ {\isachardoublequoteopen}concurrent{\isacharunderscore}ops{\isacharunderscore}commute\ ys{\isachardoublequoteclose}\isanewline
\ \ \ \ \ \ \ \ \ \ {\isachardoublequoteopen}distinct\ xs{\isachardoublequoteclose}\isanewline
\ \ \ \ \ \ \ \ \ \ {\isachardoublequoteopen}distinct\ ys{\isachardoublequoteclose}\isanewline
\ \ \ \ \ \ \ \ \ \ {\isachardoublequoteopen}hb{\isacharunderscore}consistent\ xs{\isachardoublequoteclose}\isanewline
\ \ \ \ \ \ \ \ \ \ {\isachardoublequoteopen}hb{\isacharunderscore}consistent\ ys{\isachardoublequoteclose}\isanewline
\ \ \isakeyword{shows}\ \ \ {\isachardoublequoteopen}apply{\isacharunderscore}operations\ xs\ {\isacharequal}\ apply{\isacharunderscore}operations\ ys{\isachardoublequoteclose}\isanewline
\end{isabellebody}

The proof follows by induction on $xs$. The empty list case is immediate. Assmuming the property true for $xs$, we prove for $xs@[x]$. One can split $ys$ such that
$ys = prefix@x@suffix$ and "concurrent set x suffix".
We prove that "apply oeprations xs = apply operations prefix@ suffix"
finally
"apply operations (xs@[x]) = apply operations (prefix@suffix@[x]) = apply oepratiosn (prefix@[x]@suffix)"
The entire proof is shown in Figure~\ref{fig.convergence}.


\begin{figure}
  \raggedright
  \begin{isabellebody}
\isanewline
\isacommand{theorem}\isamarkupfalse%
\ \ convergence{\isacharcolon}\isanewline
\ \ \isakeyword{assumes}\ {\isachardoublequoteopen}set\ xs\ {\isacharequal}\ set\ ys{\isachardoublequoteclose}\isanewline
\ \ \ \ \ \ \ \ \ \ {\isachardoublequoteopen}concurrent{\isacharunderscore}ops{\isacharunderscore}commute\ xs{\isachardoublequoteclose}\isanewline
\ \ \ \ \ \ \ \ \ \ {\isachardoublequoteopen}concurrent{\isacharunderscore}ops{\isacharunderscore}commute\ ys{\isachardoublequoteclose}\isanewline
\ \ \ \ \ \ \ \ \ \ {\isachardoublequoteopen}distinct\ xs{\isachardoublequoteclose}\isanewline
\ \ \ \ \ \ \ \ \ \ {\isachardoublequoteopen}distinct\ ys{\isachardoublequoteclose}\isanewline
\ \ \ \ \ \ \ \ \ \ {\isachardoublequoteopen}hb{\isacharunderscore}consistent\ xs{\isachardoublequoteclose}\isanewline
\ \ \ \ \ \ \ \ \ \ {\isachardoublequoteopen}hb{\isacharunderscore}consistent\ ys{\isachardoublequoteclose}\isanewline
\ \ \isakeyword{shows}\ \ \ {\isachardoublequoteopen}apply{\isacharunderscore}operations\ xs\ {\isacharequal}\ apply{\isacharunderscore}operations\ ys{\isachardoublequoteclose}\isanewline
%
\isacommand{using}\isamarkupfalse%
\ assms\ \isacommand{proof}\isamarkupfalse%
{\isacharparenleft}induction\ xs\ arbitrary{\isacharcolon}\ ys\ rule{\isacharcolon}\ rev{\isacharunderscore}induct{\isacharcomma}\ simp{\isacharparenright}\isanewline
\ \ \isacommand{case}\isamarkupfalse%
\ assms{\isacharcolon}\ {\isacharparenleft}snoc\ x\ xs{\isacharparenright}\isanewline
\ \ \isacommand{then}\isamarkupfalse%
\ \isacommand{obtain}\isamarkupfalse%
\ prefix\ suffix\ \isakeyword{where}\ ys{\isacharunderscore}split{\isacharcolon}\ {\isachardoublequoteopen}ys\ {\isacharequal}\ prefix\ {\isacharat}\ x\ {\isacharhash}\ suffix\ {\isasymand}\ concurrent{\isacharunderscore}set\ x\ suffix{\isachardoublequoteclose}\isanewline
\ \ \ \ \isacommand{using}\isamarkupfalse%
\ hb{\isacharunderscore}consistent{\isacharunderscore}prefix{\isacharunderscore}suffix{\isacharunderscore}exists\ \isacommand{by}\isamarkupfalse%
\ fastforce\isanewline
\ \ \isacommand{moreover}\isamarkupfalse%
\ \isacommand{hence}\isamarkupfalse%
\ {\isacharasterisk}{\isacharcolon}\ {\isachardoublequoteopen}distinct\ {\isacharparenleft}prefix\ {\isacharat}\ suffix{\isacharparenright}{\isachardoublequoteclose}\ {\isachardoublequoteopen}hb{\isacharunderscore}consistent\ xs{\isachardoublequoteclose}\isanewline
\ \ \ \ \isacommand{using}\isamarkupfalse%
\ assms\ \isacommand{by}\isamarkupfalse%
\ auto\isanewline
\ \ \isacommand{moreover}\isamarkupfalse%
\ \isacommand{{\isacharbraceleft}}\isamarkupfalse%
\isanewline
\ \ \ \ \isacommand{have}\isamarkupfalse%
\ {\isachardoublequoteopen}hb{\isacharunderscore}consistent\ prefix{\isachardoublequoteclose}\ {\isachardoublequoteopen}hb{\isacharunderscore}consistent\ suffix{\isachardoublequoteclose}\isanewline
\ \ \ \ \ \ \isacommand{using}\isamarkupfalse%
\ ys{\isacharunderscore}split\ assms\ hb{\isacharunderscore}consistent{\isacharunderscore}append{\isacharunderscore}D{\isadigit{2}}\ hb{\isacharunderscore}consistent{\isacharunderscore}append{\isacharunderscore}elim{\isacharunderscore}ConsD\ \isacommand{by}\isamarkupfalse%
\ blast{\isacharplus}\isanewline
\ \ \ \ \isacommand{hence}\isamarkupfalse%
\ {\isachardoublequoteopen}hb{\isacharunderscore}consistent\ {\isacharparenleft}prefix\ {\isacharat}\ suffix{\isacharparenright}{\isachardoublequoteclose}\isanewline
\ \ \ \ \ \ \isacommand{by}\isamarkupfalse%
\ {\isacharparenleft}metis\ assms{\isacharparenleft}{\isadigit{8}}{\isacharparenright}\ hb{\isacharunderscore}consistent{\isacharunderscore}append\ hb{\isacharunderscore}consistent{\isacharunderscore}append{\isacharunderscore}porder\ list{\isachardot}set{\isacharunderscore}intros{\isacharparenleft}{\isadigit{2}}{\isacharparenright}\ ys{\isacharunderscore}split{\isacharparenright}\isanewline
\ \ \isacommand{{\isacharbraceright}}\isamarkupfalse%
\isanewline
\ \ \isacommand{moreover}\isamarkupfalse%
\ \isacommand{have}\isamarkupfalse%
\ {\isacharasterisk}{\isacharasterisk}{\isacharcolon}\ {\isachardoublequoteopen}concurrent{\isacharunderscore}ops{\isacharunderscore}commute\ {\isacharparenleft}prefix\ {\isacharat}\ suffix\ {\isacharat}\ {\isacharbrackleft}x{\isacharbrackright}{\isacharparenright}{\isachardoublequoteclose}\isanewline
\ \ \ \ \isacommand{using}\isamarkupfalse%
\ assms\ ys{\isacharunderscore}split\ \isacommand{by}\isamarkupfalse%
\ {\isacharparenleft}clarsimp\ simp{\isacharcolon}\ concurrent{\isacharunderscore}ops{\isacharunderscore}commute{\isacharunderscore}def{\isacharparenright}\isanewline
\ \ \isacommand{moreover}\isamarkupfalse%
\ \isacommand{hence}\isamarkupfalse%
\ {\isachardoublequoteopen}concurrent{\isacharunderscore}ops{\isacharunderscore}commute\ {\isacharparenleft}prefix\ {\isacharat}\ suffix{\isacharparenright}{\isachardoublequoteclose}\isanewline
\ \ \ \ \isacommand{by}\isamarkupfalse%
\ {\isacharparenleft}force\ simp\ del{\isacharcolon}\ append{\isacharunderscore}assoc\ simp\ add{\isacharcolon}\ append{\isacharunderscore}assoc{\isacharbrackleft}symmetric{\isacharbrackright}{\isacharparenright}\isanewline
\ \ \isacommand{ultimately}\isamarkupfalse%
\ \isacommand{have}\isamarkupfalse%
\ {\isachardoublequoteopen}apply{\isacharunderscore}operations\ xs\ {\isacharequal}\ apply{\isacharunderscore}operations\ {\isacharparenleft}prefix{\isacharat}suffix{\isacharparenright}{\isachardoublequoteclose}\isanewline
\ \ \ \ \isacommand{using}\isamarkupfalse%
\ assms\ \isacommand{by}\isamarkupfalse%
\ simp\ {\isacharparenleft}metis\ Diff{\isacharunderscore}insert{\isacharunderscore}absorb\ Un{\isacharunderscore}iff\ {\isacharasterisk}\ concurrent{\isacharunderscore}ops{\isacharunderscore}commute{\isacharunderscore}appendD\ set{\isacharunderscore}append{\isacharparenright}\isanewline
\ \ \isacommand{moreover}\isamarkupfalse%
\ \isacommand{have}\isamarkupfalse%
\ {\isachardoublequoteopen}apply{\isacharunderscore}operations\ {\isacharparenleft}prefix{\isacharat}suffix\ {\isacharat}\ {\isacharbrackleft}x{\isacharbrackright}{\isacharparenright}\ {\isacharequal}\ apply{\isacharunderscore}operations\ {\isacharparenleft}prefix{\isacharat}x\ {\isacharhash}\ suffix{\isacharparenright}{\isachardoublequoteclose}\isanewline
\ \ \ \ \isacommand{using}\isamarkupfalse%
\ ys{\isacharunderscore}split\ assms\ {\isacharasterisk}{\isacharasterisk}\ concurrent{\isacharunderscore}ops{\isacharunderscore}commute{\isacharunderscore}concurrent{\isacharunderscore}set\ \isacommand{by}\isamarkupfalse%
\ force\isanewline
\ \ \isacommand{ultimately}\isamarkupfalse%
\ \isacommand{show}\isamarkupfalse%
\ {\isacharquery}case\isanewline
\ \ \ \ \isacommand{using}\isamarkupfalse%
\ ys{\isacharunderscore}split\ \isacommand{by}\isamarkupfalse%
\ {\isacharparenleft}force\ simp{\isacharcolon}\ append{\isacharunderscore}assoc{\isacharbrackleft}symmetric{\isacharbrackright}\ simp\ del{\isacharcolon}\ append{\isacharunderscore}assoc{\isacharparenright}\isanewline
\isacommand{qed}\isamarkupfalse%
  \end{isabellebody}
  \caption{Proof of convergence theorem in Isar.}
  \label{fig.converge}
\end{figure}


\section{Network}
\label{sect.network}

\section{Replicated Growable Array}
\label{sect.rga}

\section{RGA-Network}
\label{sect.rga.network}

\section{Example}
\label{sect.example}

\section{Discussion}
\label{sect.discussion}

\section{Limitations}
\label{sect.limitations}

\section{Related Work}
\label{sect.relatedwork}


\subsection{Formal Verification}

Zeller et al. have used Isabelle to formally specify and verify a number of state-based CRDTs~\cite{Zeller:2014fl}.

Some operational transformation functions have also been formally specified and verified using the
SPIKE theorem prover~\cite{Imine:2003ks,Imine:2006kn}, Coq~\cite{Sinchuk:2016cf}, and
Isabelle~\cite{Jungnickel:2015ua}. These efforts focus on proving that the transformation functions
satisfy given properties (such as the \emph{transformation properties} $\mathit{TP}_1$ and
$\mathit{TP}_2$~\cite{Oster:2006tr,Ressel:1996wx}).

control algorithm (also known as integration algorithm)

Prove commutativity properties of the transformation function, but do not formally relate these
properties to a particular network model. Informal reasoning is used to demonstrate that these
properties do indeed ensure convergence in all possible
executions~\cite{Suleiman:1998eu,Sun:1998vf}.


% \cite{Burckhardt:2014ft}
% \cite{Roh:2009ws}
% \cite{Bieniusa:2012gt}
% \cite{Wang:2015vo}
% \cite{Spiewak:2010vw}
% \cite{Google:2015vk}
% \cite{Lemonik:2016wh}
% \cite{Lamport:1978jq}
% \cite{Preguica:2012fx}
% \cite{Schwarz:1994gl}
% \cite{ParkerJr:1983jb}
% vector clocks~\cite{Fidge:1988tv,Raynal:1996jl}
% \cite{Defago:2004ji}
% \cite{Chandra:1996cp}
% \cite{Davidson:1985hv}
% \cite{Terry:1995dn}
% \cite{DeCandia:2007ui}

\section{Conclusion}
\label{sect.conclusion}

\subsection*{Acknowledgements}

\bibliography{references}{}

\end{document}
