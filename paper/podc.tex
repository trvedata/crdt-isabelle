\documentclass[11pt]{article}

\RequirePackage[letterpaper,portrait,margin=1in]{geometry}
\PassOptionsToPackage{hyphens}{url}
\usepackage[british]{babel}
\usepackage[colorlinks]{hyperref}
\usepackage[hyphenbreaks]{breakurl} % Fix URL line breaking when using dvips (e.g. arxiv.org)
\usepackage[numbers,sort]{natbib}
\usepackage{microtype}

\title{Formal Verification of Peer-to-Peer Collaborative Editing}
\author{Victor~B.~F.~Gomes, Martin Kleppmann, Dominic P.~Mulligan,\\Alastair R. Beresford}
\date{Computer Laboratory, University of Cambridge}

\begin{document}

\maketitle

\begin{abstract}
To be completed...
\end{abstract}

%%%%%%%%%%%%%%%%%%%%%%%%%%%%%%%%%%%%%%%%%%%%%%%%%%%%%%%%%%%%%%%%%%%%%%%%%%%%%%%%
% Section
%%%%%%%%%%%%%%%%%%%%%%%%%%%%%%%%%%%%%%%%%%%%%%%%%%%%%%%%%%%%%%%%%%%%%%%%%%%%%%%%

\newpage
\section{Introduction}
\label{sect.introduction}

Collaborative editing applications such as Google Docs~\cite{DayRichter:2010tt}, Microsoft Word
Online, Etherpad~\cite{Etherpad:2011um}, and Novell Vibe~\cite{Spiewak:2010vw} are increasing in
popularity. A common feature of these tools is that they allow several users to concurrently modify
a document without having to send the document back and forth (e.g., by email), and without
requiring any exclusive locking or manual resolution of merge conflicts.

However, all currently deployed collaborative editing systems rely on a central server to determine
a sequential order of edit operations, a design originally pioneered by the Jupiter
system~\cite{Nichols:1995fd}. This architecture has the advantage of simplifying the collaborative
editing algorithm by restricting the concurrency in the system, but it has the downside of placing
significant trust in that single server: it is at risk of being compromised, censored, seized, or
otherwise subverted by adversaries. For sensitive scenarios, such as communication between
dissidents of a repressive regime, such centralization is problematic.

Peer-to-peer systems with end-to-end encryption can be more robust against such interference. In
this setting, especially when the participating nodes are mobile devices, totally ordered broadcast
of editing operations is prohibitively expensive~\cite{Attiya:2015dm}. Thus, there has been
significant interest in collaborative editing algorithms that work correctly in the face of the
increased concurrency encountered in peer-to-peer systems~\cite{Randolph:2015gj}.

There two families of algorithms for collaborative editing: \emph{operational transformation}
(OT)~\cite{Ellis:1989ue,Ressel:1996wx,Oster:2006tr,Sun:1998vf,Sun:1998un,Suleiman:1998eu,Nichols:1995fd}
and \emph{conflict-free replicated datatypes}
(CRDTs)~\cite{Shapiro:2011wy,Roh:2011dw,Preguica:2009fz,Oster:2006wj,Weiss:2010hx,Nedelec:2013ky,Kleppmann:2016ve}.
Both allow a document to be modified concurrently on different replicas, with changes applied
immediately to the local copy, while asynchronously propagating changes to other replicas. The
goal of these algorithms is to ensure that for all concurrent executions, the replicas converge
toward the same state without any edits being lost (a property known as \emph{strong eventual
consistency}~\cite{Shapiro:2011un}). We discuss the relationship between OT and CRDTs further in
Section~\ref{sect.relatedwork}.

However, these algorithms have a checkered history. OT algorithms have a reputation of being very
difficult to understand and to implement correctly~\cite{Spiewak:2010vw}. Despite the fact that OT
has been studied for almost three decades~\cite{Ellis:1989ue}, few algorithms work correctly in a
peer-to-peer setting, and several published algorithms were later shown to violating their supposed
guarantees~\cite{Imine:2003ks,Imine:2006kn,Li:2004er,Randolph:2015gj}.

Of one particular algorithm called Replicated Growable Array~\cite{Roh:2011dw} (RGA, which we study
in this paper), Attiya et al. recently wrote, ``the reason why RGA actually works has been a bit of
a mystery''~\cite{Attiya:2016kh}.

The problem that makes collaborative algorithms difficult is that arbitrary edits may occur
concurrently. Since each participant may apply edits to the document in a different order, and there
are many possible interleavings of concurrent operations, informal reasoning about the correctness
of these algorithms quickly becomes unmanageable.

Formal verification can help address this problem, by encouraging us to break down the problem into
manageable modules with well-defined properties, and by using machine-checked proofs and automated
reasoning to verify that an algorithm satisfies those properties. In this paper we present a modular
framework for such formal verification of collaborative editing algorithms.

op-based vs. state-based

The main contributions of this paper are the first machine-checked correctness proof of the RGA
algorithm, of any op-based CRDT? + network model

Modular design that treats the convergence proof, CRDT, and network model separately. Reusable

The next generation is going to use CRDTs. They are more efficient, it supports p2p and e2e, but currently lacking a formal
basis. We provide the foundation for ensuring the correctness of next-generation systems.
FB/WhatsApp reaching over a billion users with e2e encryption.

CRDTs used in industry


\section{Related Work}
\label{sect.relatedwork}

Previously published correctness proofs~\cite{Attiya:2016kh,Kleppmann:2016ve,Roh:2009ws} of RGA have
used informal natural language reasoning; to our knowledge, ours is the first mechanized proof of
the algorithm.

The main high-level difference between these families is that CRDTs define operations such that
concurrent operations are commutative by design; by contrast, OT allows non-commutative operations,
and instead defines transformation functions that rewrite operations so that they have the same
effect on every replica, even if the operations are applied in different orders.

\subsection{Formal Verification}

Zeller et al. have used Isabelle to formally specify and verify a number of state-based CRDTs~\cite{Zeller:2014fl}.

Some operational transformation functions have also been formally specified and verified using the
SPIKE theorem prover~\cite{Imine:2003ks,Imine:2006kn}, Coq~\cite{Sinchuk:2016cf}, and
Isabelle~\cite{Jungnickel:2015ua}. These efforts focus on proving that the transformation functions
satisfy given properties (such as the \emph{transformation properties} $\mathit{TP}_1$ and
$\mathit{TP}_2$~\cite{Oster:2006tr,Ressel:1996wx}).

control algorithm (also known as integration algorithm)

Prove commutativity properties of the transformation function, but do not formally relate these
properties to a particular network model. Informal reasoning is used to demonstrate that these
properties do indeed ensure convergence in all possible
executions~\cite{Suleiman:1998eu,Sun:1998vf}.


% \cite{Burckhardt:2014ft}
% \cite{Baquero:2014ed}
% \cite{Roh:2009ws}
% \cite{Bieniusa:2012gt}
% \cite{Almeida:2016tk}
% \cite{Baquero:2015tm}
% \cite{Li:2006kd}
% \cite{Wang:2015vo}
% \cite{Spiewak:2010vw}
% \cite{Google:2015vk}
% \cite{Lemonik:2016wh}
% \cite{Lamport:1978jq}
% \cite{Preguica:2012fx}
% \cite{Schwarz:1994gl}
% \cite{ParkerJr:1983jb}
% vector clocks~\cite{Fidge:1988tv,Raynal:1996jl}
% \cite{Defago:2004ji}
% \cite{Chandra:1996cp}
% \cite{Davidson:1985hv}
% \cite{Terry:1995dn}
% \cite{DeCandia:2007ui}

\section{Conclusion}
\label{sect.conclusion}

\subsection*{Acknowledgements}

\bibliographystyle{plainnat}
\bibliography{references}{}

\end{document}
